% Options for packages loaded elsewhere
\PassOptionsToPackage{unicode}{hyperref}
\PassOptionsToPackage{hyphens}{url}
%
\documentclass[
]{article}
\usepackage{amsmath,amssymb}
\usepackage{lmodern}
\usepackage{ifxetex,ifluatex}
\ifnum 0\ifxetex 1\fi\ifluatex 1\fi=0 % if pdftex
  \usepackage[T1]{fontenc}
  \usepackage[utf8]{inputenc}
  \usepackage{textcomp} % provide euro and other symbols
\else % if luatex or xetex
  \usepackage{unicode-math}
  \defaultfontfeatures{Scale=MatchLowercase}
  \defaultfontfeatures[\rmfamily]{Ligatures=TeX,Scale=1}
\fi
% Use upquote if available, for straight quotes in verbatim environments
\IfFileExists{upquote.sty}{\usepackage{upquote}}{}
\IfFileExists{microtype.sty}{% use microtype if available
  \usepackage[]{microtype}
  \UseMicrotypeSet[protrusion]{basicmath} % disable protrusion for tt fonts
}{}
\makeatletter
\@ifundefined{KOMAClassName}{% if non-KOMA class
  \IfFileExists{parskip.sty}{%
    \usepackage{parskip}
  }{% else
    \setlength{\parindent}{0pt}
    \setlength{\parskip}{6pt plus 2pt minus 1pt}}
}{% if KOMA class
  \KOMAoptions{parskip=half}}
\makeatother
\usepackage{xcolor}
\IfFileExists{xurl.sty}{\usepackage{xurl}}{} % add URL line breaks if available
\IfFileExists{bookmark.sty}{\usepackage{bookmark}}{\usepackage{hyperref}}
\hypersetup{
  pdftitle={MATH 3080 Lab Project 2},
  pdfauthor={Harrison Webb},
  hidelinks,
  pdfcreator={LaTeX via pandoc}}
\urlstyle{same} % disable monospaced font for URLs
\usepackage[margin=1in]{geometry}
\usepackage{color}
\usepackage{fancyvrb}
\newcommand{\VerbBar}{|}
\newcommand{\VERB}{\Verb[commandchars=\\\{\}]}
\DefineVerbatimEnvironment{Highlighting}{Verbatim}{commandchars=\\\{\}}
% Add ',fontsize=\small' for more characters per line
\usepackage{framed}
\definecolor{shadecolor}{RGB}{248,248,248}
\newenvironment{Shaded}{\begin{snugshade}}{\end{snugshade}}
\newcommand{\AlertTok}[1]{\textcolor[rgb]{0.94,0.16,0.16}{#1}}
\newcommand{\AnnotationTok}[1]{\textcolor[rgb]{0.56,0.35,0.01}{\textbf{\textit{#1}}}}
\newcommand{\AttributeTok}[1]{\textcolor[rgb]{0.77,0.63,0.00}{#1}}
\newcommand{\BaseNTok}[1]{\textcolor[rgb]{0.00,0.00,0.81}{#1}}
\newcommand{\BuiltInTok}[1]{#1}
\newcommand{\CharTok}[1]{\textcolor[rgb]{0.31,0.60,0.02}{#1}}
\newcommand{\CommentTok}[1]{\textcolor[rgb]{0.56,0.35,0.01}{\textit{#1}}}
\newcommand{\CommentVarTok}[1]{\textcolor[rgb]{0.56,0.35,0.01}{\textbf{\textit{#1}}}}
\newcommand{\ConstantTok}[1]{\textcolor[rgb]{0.00,0.00,0.00}{#1}}
\newcommand{\ControlFlowTok}[1]{\textcolor[rgb]{0.13,0.29,0.53}{\textbf{#1}}}
\newcommand{\DataTypeTok}[1]{\textcolor[rgb]{0.13,0.29,0.53}{#1}}
\newcommand{\DecValTok}[1]{\textcolor[rgb]{0.00,0.00,0.81}{#1}}
\newcommand{\DocumentationTok}[1]{\textcolor[rgb]{0.56,0.35,0.01}{\textbf{\textit{#1}}}}
\newcommand{\ErrorTok}[1]{\textcolor[rgb]{0.64,0.00,0.00}{\textbf{#1}}}
\newcommand{\ExtensionTok}[1]{#1}
\newcommand{\FloatTok}[1]{\textcolor[rgb]{0.00,0.00,0.81}{#1}}
\newcommand{\FunctionTok}[1]{\textcolor[rgb]{0.00,0.00,0.00}{#1}}
\newcommand{\ImportTok}[1]{#1}
\newcommand{\InformationTok}[1]{\textcolor[rgb]{0.56,0.35,0.01}{\textbf{\textit{#1}}}}
\newcommand{\KeywordTok}[1]{\textcolor[rgb]{0.13,0.29,0.53}{\textbf{#1}}}
\newcommand{\NormalTok}[1]{#1}
\newcommand{\OperatorTok}[1]{\textcolor[rgb]{0.81,0.36,0.00}{\textbf{#1}}}
\newcommand{\OtherTok}[1]{\textcolor[rgb]{0.56,0.35,0.01}{#1}}
\newcommand{\PreprocessorTok}[1]{\textcolor[rgb]{0.56,0.35,0.01}{\textit{#1}}}
\newcommand{\RegionMarkerTok}[1]{#1}
\newcommand{\SpecialCharTok}[1]{\textcolor[rgb]{0.00,0.00,0.00}{#1}}
\newcommand{\SpecialStringTok}[1]{\textcolor[rgb]{0.31,0.60,0.02}{#1}}
\newcommand{\StringTok}[1]{\textcolor[rgb]{0.31,0.60,0.02}{#1}}
\newcommand{\VariableTok}[1]{\textcolor[rgb]{0.00,0.00,0.00}{#1}}
\newcommand{\VerbatimStringTok}[1]{\textcolor[rgb]{0.31,0.60,0.02}{#1}}
\newcommand{\WarningTok}[1]{\textcolor[rgb]{0.56,0.35,0.01}{\textbf{\textit{#1}}}}
\usepackage{graphicx}
\makeatletter
\def\maxwidth{\ifdim\Gin@nat@width>\linewidth\linewidth\else\Gin@nat@width\fi}
\def\maxheight{\ifdim\Gin@nat@height>\textheight\textheight\else\Gin@nat@height\fi}
\makeatother
% Scale images if necessary, so that they will not overflow the page
% margins by default, and it is still possible to overwrite the defaults
% using explicit options in \includegraphics[width, height, ...]{}
\setkeys{Gin}{width=\maxwidth,height=\maxheight,keepaspectratio}
% Set default figure placement to htbp
\makeatletter
\def\fps@figure{htbp}
\makeatother
\setlength{\emergencystretch}{3em} % prevent overfull lines
\providecommand{\tightlist}{%
  \setlength{\itemsep}{0pt}\setlength{\parskip}{0pt}}
\setcounter{secnumdepth}{-\maxdimen} % remove section numbering
\ifluatex
  \usepackage{selnolig}  % disable illegal ligatures
\fi

\title{MATH 3080 Lab Project 2}
\author{Harrison Webb}
\date{1/20/2022}

\begin{document}
\maketitle

{
\setcounter{tocdepth}{2}
\tableofcontents
}
\hypertarget{problem-1}{%
\section{Problem 1}\label{problem-1}}

\emph{Write a function that takes a variable number of arguments and
prints them with each of the inputs separated by a new line. (Hint:
There is a special character, \texttt{\textbackslash{}n}, that
represents new lines. The following code should suggest what to do:
\texttt{cat("hello",\ "awesome",\ "world",\ sep\ =\ "\textbackslash{}n")})}

\begin{Shaded}
\begin{Highlighting}[]
\NormalTok{printArguments }\OtherTok{=} \ControlFlowTok{function}\NormalTok{(...) \{}
  \FunctionTok{cat}\NormalTok{(}\FunctionTok{c}\NormalTok{(...), }\AttributeTok{sep =} \StringTok{"}\SpecialCharTok{\textbackslash{}n}\StringTok{"}\NormalTok{)}
\NormalTok{\}}

\FunctionTok{printArguments}\NormalTok{(}\StringTok{"hi"}\NormalTok{, }\StringTok{"hello"}\NormalTok{, }\StringTok{"hola"}\NormalTok{, }\DecValTok{1}\NormalTok{, }\DecValTok{2}\NormalTok{, }\DecValTok{3}\NormalTok{)}
\end{Highlighting}
\end{Shaded}

\begin{verbatim}
## hi
## hello
## hola
## 1
## 2
## 3
\end{verbatim}

\hypertarget{problem-2}{%
\section{Problem 2}\label{problem-2}}

\emph{Write an infix operator that represents logical XOR. In logic,
\texttt{x\ xor\ y} is true if only one of either \texttt{x} or
\texttt{y} are true; if neither are true or both are true, then it's
false. The following function implements XOR:}

\begin{Shaded}
\begin{Highlighting}[]
\NormalTok{xor }\OtherTok{\textless{}{-}} \ControlFlowTok{function}\NormalTok{(x, y) \{}
\NormalTok{  (x }\SpecialCharTok{|}\NormalTok{ y) }\SpecialCharTok{\&}\NormalTok{ (}\SpecialCharTok{!}\NormalTok{x }\SpecialCharTok{|} \SpecialCharTok{!}\NormalTok{y)}
\NormalTok{\}}
\end{Highlighting}
\end{Shaded}

\emph{Write the infix operator \texttt{\%xor\%} that allows for the
syntax \texttt{x\ \%xor\%\ y}.}

\begin{Shaded}
\begin{Highlighting}[]
\CommentTok{\# Your code here}
\StringTok{\textasciigrave{}}\AttributeTok{\%xor\%}\StringTok{\textasciigrave{}} \OtherTok{\textless{}{-}} \ControlFlowTok{function}\NormalTok{(x, y) \{}\FunctionTok{xor}\NormalTok{(x,y)\} }\CommentTok{\#just reuse above xor function}

\ConstantTok{FALSE} \SpecialCharTok{\%xor\%} \ConstantTok{FALSE}
\end{Highlighting}
\end{Shaded}

\begin{verbatim}
## [1] FALSE
\end{verbatim}

\begin{Shaded}
\begin{Highlighting}[]
\ConstantTok{TRUE} \SpecialCharTok{\%xor\%} \ConstantTok{FALSE}
\end{Highlighting}
\end{Shaded}

\begin{verbatim}
## [1] TRUE
\end{verbatim}

\begin{Shaded}
\begin{Highlighting}[]
\ConstantTok{TRUE} \SpecialCharTok{\%xor\%} \ConstantTok{TRUE}
\end{Highlighting}
\end{Shaded}

\begin{verbatim}
## [1] FALSE
\end{verbatim}

\emph{The following should work as anticipated:}

\begin{Shaded}
\begin{Highlighting}[]
\ConstantTok{TRUE} \SpecialCharTok{\%xor\%} \ConstantTok{TRUE}  \CommentTok{\# Should be FALSE}
\end{Highlighting}
\end{Shaded}

\begin{verbatim}
## [1] FALSE
\end{verbatim}

\begin{Shaded}
\begin{Highlighting}[]
\ConstantTok{FALSE} \SpecialCharTok{\%xor\%} \ConstantTok{TRUE}  \CommentTok{\# Should be TRUE}
\end{Highlighting}
\end{Shaded}

\begin{verbatim}
## [1] TRUE
\end{verbatim}

\begin{Shaded}
\begin{Highlighting}[]
\ConstantTok{TRUE} \SpecialCharTok{\%xor\%} \ConstantTok{FALSE}  \CommentTok{\# Should be TRUE}
\end{Highlighting}
\end{Shaded}

\begin{verbatim}
## [1] TRUE
\end{verbatim}

\begin{Shaded}
\begin{Highlighting}[]
\ConstantTok{FALSE} \SpecialCharTok{\%xor\%} \ConstantTok{FALSE}  \CommentTok{\# Should be FALSE}
\end{Highlighting}
\end{Shaded}

\begin{verbatim}
## [1] FALSE
\end{verbatim}

\hypertarget{problem-3}{%
\section{Problem 3}\label{problem-3}}

\emph{Newton's method is a numerical root-finding technique; that is,
given a function \(f\), the objective of the method is to find an input
\(x\) such that \(f(x) = 0\). We call such an \(x\) a} root. \emph{The
method is iterative. We start with an initial guess \(x_0\). The
algorithm then produces new approximations for the root \(x\) via the
formula:}

\[x_{n + 1} = x_n - \frac{f(x_n)}{f'(x_n)}.\]

\emph{We need a rule for stopping the algorithm, and we could either
stop at some fixed \(N\) or when
\(\left|x_{n+1} - x_n\right| < \epsilon\) for some user-selected
\(\epsilon > 0\). (This represents some tolerable numerical error.)}

\emph{In this project you will write a function implementing Newton's
method; call the function \texttt{newton\_solver()}. Based on the above
description this function must take at least the following inputs:}

\begin{itemize}
\tightlist
\item
  \emph{An initial \(x_0\);}
\item
  \emph{A function \(f\);}
\item
  \emph{The function's derivative \(f'\);}
\item
  \emph{A maximum number of iterations \(N\); and}
\item
  \emph{A desired numerical tolerance \(\epsilon\).}
\end{itemize}

\emph{(One may think we need either \(\epsilon\) or \(N\) but in
practice we should always have \(N\) to ensure the algorithm
terminates.)}

\emph{We will add additional behavioral constraints to the function.}

\begin{itemize}
\tightlist
\item
  \emph{There will be a loop where the update algorithm is applied. This
  loop should terminate immediately if the numerical tolerance threshold
  is met; this can be achieved via an \texttt{if} statement and
  \texttt{break}. But if the loop hits \(N\) iterations, the function
  should throw a warning.}
\item
  \emph{\(f\) and \(f'\) should be functions. They should return
  univariate \texttt{numeric} values. If there ever comes a time where
  the input functions don't return a single number, then
  \texttt{newton\_solver()} should throw an error.}
\item
  \emph{It's possible that \(f'(x_n)\) could become zero and then a
  division-by-zero error will occur. \texttt{newton\_solver()} should
  stop with an error informing the user that the derivative became
  zero.}
\item
  \emph{We could have our function return a list with detailed
  information not just with the obtained root but also with the value of
  \(f\) at the root or how many iterations of the algorithm went
  through. But instead, we will just have the function return the
  obtained root.}
\item
  \emph{The maximum number of iterations \(N\) should be a positive
  number; the same should be said for \(\epsilon\). If not, an error
  should be thrown.}
\end{itemize}

\begin{enumerate}
\def\labelenumi{\arabic{enumi}.}
\tightlist
\item
  \emph{Write \texttt{newton\_solver()} based on the description above.}
\end{enumerate}

\begin{Shaded}
\begin{Highlighting}[]
\CommentTok{\#\textquotesingle{} Newton\textquotesingle{}s Method for Finding Roots}
\CommentTok{\#\textquotesingle{}}
\CommentTok{\#\textquotesingle{} Implements Newton\textquotesingle{}s method for finding roots of functions numerically}
\CommentTok{\#\textquotesingle{}}
\CommentTok{\#\textquotesingle{} This function implements Newton\textquotesingle{}s method, a numerical root finding technique.}
\CommentTok{\#\textquotesingle{} Given a function \textbackslash{}eqn\{f\}, its derivative \textbackslash{}eqn\{f\textquotesingle{}\}, and an initial guess for}
\CommentTok{\#\textquotesingle{} the root \textbackslash{}eqn\{x\_0\}, the function finds the root via the iterative formula}
\CommentTok{\#\textquotesingle{}}
\CommentTok{\#\textquotesingle{} \textbackslash{}deqn\{x\_\{n+1\} = x\_n {-} \textbackslash{}frac\{f(x\_n)\}\{f\textquotesingle{}(x\_n)\}\}}
\CommentTok{\#\textquotesingle{}}
\CommentTok{\#\textquotesingle{} @param f The function for which a root is sought}
\CommentTok{\#\textquotesingle{} @param fprime The function representing the derivative of \textbackslash{}code\{f\}}
\CommentTok{\#\textquotesingle{} @param x0 The initial guess of the root}
\CommentTok{\#\textquotesingle{} @param N The maximum number of iterations}
\CommentTok{\#\textquotesingle{} @param eps The tolerable numerical error \textbackslash{}eqn\{\textbackslash{}epsilon\}}
\CommentTok{\#\textquotesingle{} @return The root of \textbackslash{}code\{f\}}
\CommentTok{\#\textquotesingle{} @examples}
\CommentTok{\#\textquotesingle{} f \textless{}{-} function(x) \{x\^{}2\}}
\CommentTok{\#\textquotesingle{} fprime \textless{}{-} function(x) \{2 * x\}}
\CommentTok{\#\textquotesingle{} newton\_solver(f, fprime, x0 = 10, eps = 10\^{}({-}4))}
\NormalTok{newton\_solver }\OtherTok{\textless{}{-}} \ControlFlowTok{function}\NormalTok{(f, fprime, x0, }\AttributeTok{N =} \DecValTok{1000}\NormalTok{, }\AttributeTok{eps =} \FloatTok{10e{-}4}\NormalTok{) \{}
  \CommentTok{\# Your code here}
\NormalTok{\}}
\end{Highlighting}
\end{Shaded}

\emph{The following code tests whether \texttt{newton\_solver()} works
as specified. \textbf{BEWARE: IF THIS CODE DOES NOT RUN AS ANTICIPATED
OR TAKES LONGER THAN 10 SECONDS TO RUN, YOU WON'T RECEIVE CREDIT!}}

\begin{Shaded}
\begin{Highlighting}[]
\CommentTok{\# Test code for newton\_solver(); DO NOT EDIT}
\NormalTok{f1 }\OtherTok{\textless{}{-}} \ControlFlowTok{function}\NormalTok{(x) \{x}\SpecialCharTok{\^{}}\DecValTok{2}\NormalTok{\}}
\NormalTok{fprime1 }\OtherTok{\textless{}{-}} \ControlFlowTok{function}\NormalTok{(x) \{}\DecValTok{2} \SpecialCharTok{*}\NormalTok{ x\}}
\NormalTok{fprime2 }\OtherTok{\textless{}{-}} \ControlFlowTok{function}\NormalTok{(x) \{}\DecValTok{0}\NormalTok{\}}
\NormalTok{fprime3 }\OtherTok{\textless{}{-}} \ControlFlowTok{function}\NormalTok{(x) \{}\DecValTok{1}\NormalTok{\}}
\NormalTok{f2 }\OtherTok{\textless{}{-}} \ControlFlowTok{function}\NormalTok{(x) \{}\FunctionTok{c}\NormalTok{(}\DecValTok{1}\NormalTok{, x}\SpecialCharTok{\^{}}\DecValTok{2}\NormalTok{)\}}
\NormalTok{fprime4 }\OtherTok{\textless{}{-}} \ControlFlowTok{function}\NormalTok{(x) \{}\FunctionTok{c}\NormalTok{(}\DecValTok{1}\NormalTok{, }\DecValTok{2} \SpecialCharTok{*}\NormalTok{ x)\}}
\NormalTok{f3 }\OtherTok{\textless{}{-}} \ControlFlowTok{function}\NormalTok{(x) \{}\StringTok{"oopsie!"}\NormalTok{\}}
\NormalTok{fprime5 }\OtherTok{\textless{}{-}} \ControlFlowTok{function}\NormalTok{(x) \{}\StringTok{"dasies!"}\NormalTok{\}}

\CommentTok{\# The following should execute without warning or error}
\FunctionTok{newton\_solver}\NormalTok{(f1, fprime1, }\AttributeTok{x0 =} \DecValTok{10}\NormalTok{, }\AttributeTok{eps =} \FloatTok{10e{-}4}\NormalTok{)   }\CommentTok{\# Should be close to zero}
\end{Highlighting}
\end{Shaded}

\begin{verbatim}
## NULL
\end{verbatim}

\begin{Shaded}
\begin{Highlighting}[]
\FunctionTok{newton\_solver}\NormalTok{(f1, fprime1, }\AttributeTok{x0 =} \SpecialCharTok{{-}}\DecValTok{10}\NormalTok{, }\AttributeTok{eps =} \FloatTok{10e{-}4}\NormalTok{)  }\CommentTok{\# Should be close to zero}
\end{Highlighting}
\end{Shaded}

\begin{verbatim}
## NULL
\end{verbatim}

\begin{Shaded}
\begin{Highlighting}[]
\CommentTok{\# The following code should produce errors if the function was written correctly}
\FunctionTok{newton\_solver}\NormalTok{(f1, fprime2, }\AttributeTok{x0 =} \DecValTok{10}\NormalTok{)}
\end{Highlighting}
\end{Shaded}

\begin{verbatim}
## NULL
\end{verbatim}

\begin{Shaded}
\begin{Highlighting}[]
\FunctionTok{newton\_solver}\NormalTok{(f2, fprime1, }\AttributeTok{x0 =} \DecValTok{10}\NormalTok{)}
\end{Highlighting}
\end{Shaded}

\begin{verbatim}
## NULL
\end{verbatim}

\begin{Shaded}
\begin{Highlighting}[]
\FunctionTok{newton\_solver}\NormalTok{(f1, fprime4, }\AttributeTok{x0 =} \DecValTok{10}\NormalTok{)}
\end{Highlighting}
\end{Shaded}

\begin{verbatim}
## NULL
\end{verbatim}

\begin{Shaded}
\begin{Highlighting}[]
\FunctionTok{newton\_solver}\NormalTok{(f3, fprime1, }\AttributeTok{x0 =} \DecValTok{10}\NormalTok{)}
\end{Highlighting}
\end{Shaded}

\begin{verbatim}
## NULL
\end{verbatim}

\begin{Shaded}
\begin{Highlighting}[]
\FunctionTok{newton\_solver}\NormalTok{(f1, fprime5, }\AttributeTok{x0 =} \DecValTok{10}\NormalTok{)}
\end{Highlighting}
\end{Shaded}

\begin{verbatim}
## NULL
\end{verbatim}

\begin{Shaded}
\begin{Highlighting}[]
\CommentTok{\# The following code should produce warnings if the function was written}
\CommentTok{\# correctly}
\FunctionTok{newton\_solver}\NormalTok{(f1, fprime3, }\AttributeTok{x0 =} \DecValTok{10}\NormalTok{, }\AttributeTok{N =} \DecValTok{2}\NormalTok{)}
\end{Highlighting}
\end{Shaded}

\begin{verbatim}
## NULL
\end{verbatim}

\begin{enumerate}
\def\labelenumi{\arabic{enumi}.}
\setcounter{enumi}{1}
\tightlist
\item
  \emph{Use \texttt{newton\_solver()} to maximize the function
  \(g(x) = 1 - x^2\). Simple calculus should reveal that the maximum is
  \(g(0) = 1\). Maximizing \(g\) requires finding a root for \(g'\),
  since the maxima/minima of differentiable functions occurs where
  \(g'(x) = 0\). Compare the answer obtained by
  \texttt{newton\_solver()} to the known analytical result.}
\end{enumerate}

\begin{Shaded}
\begin{Highlighting}[]
\CommentTok{\# Your code here}
\end{Highlighting}
\end{Shaded}

\begin{enumerate}
\def\labelenumi{\arabic{enumi}.}
\setcounter{enumi}{2}
\tightlist
\item
  \emph{Use \texttt{newton\_solver()} to solve the equation:}
\end{enumerate}

\[e^x = -x\]

\emph{(This equation doesn't have a known analytical solution.)}

\begin{Shaded}
\begin{Highlighting}[]
\CommentTok{\# Your code here}
\end{Highlighting}
\end{Shaded}


\end{document}
